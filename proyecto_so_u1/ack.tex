\chapter{Introducción}
\label{chap:intro}

El avance y la evolución de los sistemas operativos han sido claves en la historia de las tecnologías de la información. Desde los programas primitivos que se desarrollaron para las computadoras centrales, hasta los sistemas preferidos de las computadoras personales, los dispositivos móviles y equipos integrados, los sistemas operativos han funcionado como mediadores entre lo que el usuario necesita y la complejidad del hardware \cite{ritchie1974,silberschatz2018,tanenbaum2015}. No solamente ayuda a comprender el funcionamiento de una computadora a un nivel básico el estudio de los sistemas operativos, sino que además proporciona las bases conceptuales y técnicas para desarrollar nuevas soluciones informáticas. El propósito de este documento es proporcionar una visión integral de los sistemas operativos, analizando sus características esenciales, su arquitectura y sus componentes \cite{silberschatz2018}.

\section{Objetivo de la Investigación}

El propósito fundamental de este estudio es llevar a cabo una evaluación técnica y comparativa de varios sistemas operativos, con el objetivo de determinar sus rasgos más destacados y analizar las diferentes perspectivas utilizadas en su concepción e implementación. Se examinarán arquitecturas clásicas, como la monolítica, además de opciones más actuales, como los exokernels, híbridos y microkernels \cite{engler1995}. Asimismo, se busca proporcionar una perspectiva crítica acerca de las ventajas, los límites y los campos de uso de cada sistema operativo \cite{tanenbaum2015}.

\section{Alcance de análisis}
\label{rencesec:xrefs}

El análisis incluirá, por lo menos, diez sistemas operativos que se consideran relevantes debido a su trascendencia histórica, su importancia en el ámbito académico o su utilización práctica en la industria. Dentro de ellos estarán:

\begin{itemize}
    \item \textbf{Sistemas de propósito general:} BSD, Windows, Linux. \cite{freebsd-2024,windowsinternals-2022,torvalds1991}
    \item \textbf{Sistemas educativos o académicos:} xv6, MINIX. \cite{xv6-2022,minix-docs}
    \item \textbf{Sistemas experimentales o de investigación:} Redox OS, SerenityOS. \cite{redox-2024,serenity-2024}
    \item \textbf{Sistemas de tiempo real o embebidos:} FreeRTOS. \cite{freertos-2023}
\end{itemize}

No solo se compararán las características técnicas de su arquitectura, sino también otros criterios como la comunidad de usuarios y desarrolladores, el soporte de hardware, la documentación disponible y la sostenibilidad a lo largo del tiempo \cite{silberschatz2018,freebsd-2024}. 

Así, el análisis tiene como objetivo ir más allá de una simple descripción y proporcionar una perspectiva comparativa y crítica.

\section{Metodología}

Para lograr las metas establecidas, se implementará un método que consta de tres etapas principales:

\begin{enumerate}
    \item \textbf{Revisión de libros y documentos:} Se revisarán artículos académicos, libros especializados, blogs técnicos, documentación oficial y repositorios de los sistemas operativos estudiados \cite{silberschatz2018,tanenbaum2015,ritchie1974}.
    
    \item \textbf{Análisis técnico:} Se analizarán los elementos más importantes de cada sistema, tales como la gestión de procesos, la administración de memoria, los sistemas de archivos, la gestión de dispositivos, las interfaces del usuario y los mecanismos de seguridad \cite{silberschatz2018,windowsinternals-2022}.
    
    \item \textbf{Estudio comparativo:} Se creará una tabla que resume los rasgos más relevantes de cada sistema operativo, considerando criterios como el tamaño del kernel, la arquitectura, la comunidad, el lenguaje de implementación, la documentación y el soporte de hardware \cite{engler1995,freebsd-2024,redox-2024}.
\end{enumerate}

Esta perspectiva asegurará que el análisis sea imparcial, exhaustivo y de cierta manera más provechoso para propósitos académicos y prácticos.