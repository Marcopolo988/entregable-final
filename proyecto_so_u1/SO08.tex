\section{Haiku}

\begin{figure}[H]
    \centering
    \includegraphics[width=0.35\textwidth]{figures/haiku.jpeg}
    \caption[Icono de S.O. de Haiku]%
            {Icono de S.O. de Haiku \citep{haikuos2025}}
    \label{fig:haiku}
\end{figure}

Haiku es un sistema operativo de código abierto inspirado en BeOS, iniciado en 2001 con el objetivo de proporcionar un entorno moderno, rápido y sencillo de usar, orientado principalmente a equipos de escritorio \citep{haiku2025}.  

Su arquitectura es híbrida, ya que utiliza un núcleo monolítico modular, pero con elementos que recuerdan a un microkernel, lo que equilibra rendimiento y flexibilidad \citep{haikudevdocs}. El sistema está desarrollado principalmente en C++, con partes en C y ensamblador para las interacciones de bajo nivel \citep{haikudevdocs}.  

Entre sus componentes clave se incluyen un modelo de multitarea preventiva con planificación por prioridades, gestión de memoria virtual con paginación y aislamiento de procesos, y un sistema de archivos propio llamado OpenBFS, optimizado para indexación rápida \citep{haikufs}. Además, cuenta con una interfaz gráfica uniforme gestionada por el app\_server.  

Haiku es mantenido por la Haiku Project Foundation y su comunidad de desarrolladores voluntarios. El proyecto dispone de documentación oficial, foros y repositorios activos en GitHub, con lanzamientos beta periódicos que permiten probar y evaluar su madurez \citep{haikucommunity}.
