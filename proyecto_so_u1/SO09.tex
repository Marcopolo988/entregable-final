\section{Windows}

\begin{figure}[H]
    \centering
    \includegraphics[width=0.4\textwidth]{figures/logoWindows.png}
    \caption[Ícono de Windows]
            {Ícono de Windows \citep{windowslogo2021}}
    \label{fig:windows}
\end{figure}

Windows es un sistema operativo desarrollado por Microsoft, cuyo origen se remonta a 1985 con la versión Windows 1.0, como una interfaz gráfica que extendía las funciones de MS-DOS. Desde entonces ha evolucionado hasta convertirse en una plataforma dominante en PCs, con múltiples versiones comerciales orientadas a usuarios domésticos, negocios y servidores \citep{Natividad2023}.

Windows posee una arquitectura monolítica, ya que gran parte del núcleo y los controladores E/S del dispositivo se procesan  en modo kernel compartiendo la memoria. Windows es similar al sistema de UNIX, Windows tiene incluido una abstracción de hardware (HAL) y subsistema en modo usuario con el Win 32, POSIX, OS/2. \citep{ Russinovich2005}

Está desarrollado en el lenguaje de programación C con sus componentes de bajo nivel como gestor de memoria, procesos, administración de E/S y partes de la interfaz del usuario y del sistema se implementaron en C++ y en ensamblador para las rutinas más sensibles, como arranque, gestión de interrupciones y entre otros \citep{ Russinovich2005}.

Entre sus componentes clave se encuentran el planificador de procesos y subprocesos que maneja procesos e hilos en modo kernel, el gestor de memoria virtual con paginación que protege cada proceso, subsistemas de archivos como NTFS, soporte de red integral, sistema de drivers para dispositivos hardware, interfaz gráfica de usuario (GUI), componentes de seguridad como control de acceso, y el HAL para manejar diferencias entre arquitecturas de hardware \citep{Forrester2000}.

Windows tiene una comunidad muy grande de usuarios, desarrolladores y empresas. La documentación se publica mediante Microsoft Docs, los libros “Windows Internals” son referencias reconocidas, y Microsoft suele liberar guías técnicas, SDKs y APIs para desarrolladores. Sin embargo, el código fuente no es completamente abierto, lo que limita análisis y ediciones o posibles cambios en el codigo.