\chapter*{Introduccion}

El desarrollo y la transformación de los sistemas operativos han sido fundamentales en la evolución de las tecnologías de la información. Desde los primeros programas creados para las computadoras centrales hasta los sistemas más utilizados en ordenadores personales, dispositivos móviles y equipos embebidos, los sistemas operativos han actuado como un puente entre las necesidades del usuario y la complejidad del hardware. Estudiarlos no solo permite entender el funcionamiento básico de una computadora, sino que también brinda las bases conceptuales y técnicas necesarias para el diseño de nuevas soluciones informáticas. Este documento tiene como finalidad ofrecer una perspectiva global de los sistemas operativos, abordando sus características principales, su arquitectura y sus componentes.

\section*{Objetivo de la investigacion}

El objetivo principal de este trabajo es realizar un análisis técnico y comparativo de distintos sistemas operativos, con la finalidad de identificar sus características más relevantes y examinar las diversas concepciones y enfoques aplicados en su diseño e implementación. Se estudiarán tanto arquitecturas tradicionales, como la monolítica, así como modelos más modernos, entre ellos los exokernels, híbridos y microkernels . Del mismo modo, se pretende ofrecer una visión crítica sobre las ventajas, limitaciones y ámbitos de aplicación de cada sistema operativo.

\section*{Metodologia}

Con el fin de alcanzar los objetivos planteados, se seguirá una metodología compuesta por tres fases fundamentales:

\begin{enumerate}
\item \textbf{Revisión bibliográfica y documental:} Se consultarán libros especializados, artículos académicos, blogs técnicos, documentación oficial y repositorios relacionados con los sistemas operativos en estudio.

\item \textbf{Evaluación técnica:} Se examinarán los aspectos esenciales de cada sistema, incluyendo la gestión de procesos, la administración de memoria, los sistemas de archivos, el manejo de dispositivos, las interfaces de usuario y los mecanismos de seguridad.  

\item \textbf{Comparación estructurada:} Se elaborará una tabla que sintetice las características más representativas de los sistemas operativos analizados, tomando en cuenta factores como el tamaño del kernel, la arquitectura, la comunidad, el lenguaje de implementación, la documentación disponible y el soporte de hardware.  


\end{enumerate}

Este enfoque permitirá garantizar un análisis más objetivo, completo y útil, tanto para fines académicos como prácticos.

