\section{¿Que es un sistema operativo?}
Según Stallings, el sistema operativo es el software encargado de controlar 
la ejecución de los programas y de administrar los recursos del procesador; 
además actúa como intermediario entre el usuario y el hardware, proporcionando
los servicios necesarios para que las aplicaciones puedan ejecutarse correctamente \citep{Stallings2023} 

Añadiendo esta idea, Tanenbaum y Bos explican que la finalidad del sistema 
operativo es convertir las interfaces de hardware, que suelen ser complejas e 
inconsistentes, en abstracciones practicas y sencillas de usar para los programadores
y aplicaciones. Ademas de gestionar recursos como CPU, memoria y dispositivos de E/S, permitiendo así
que las aplicaciones trabajen sin problemas con el hardware \citep{Tanenbaum2023}.

Según el artículo de la Revista Ogma, que es un científica y multidisciplinaria, un 
sistema operativo es el software principal que actúa como intermediario 
entre el usuario y el hardware, cuyo propósito es favorecer el uso eficiente 
de la computadora. Transforma las interfaces de hardware, complejas e inconsistentes, en abstracciones útiles y manejables para 
y aplicaciones y programadores, administra y mejora recursos (CPU, memoria, 
dispositivos de E/S y almacenamiento); organiza la ejecución concurrente de 
tareas, y ofrece espacios accesibles que permite el acceso de las 
computadoras, permitiendo que usuarios sin preparación instalen 
programas, administren archivos. \citep{CusmeVera2022}

Un sistema operativo es como el sistema nervioso del computador, coordinando 
componentes hardware y software para que todo funcione de manera ordenada. Se 
encarga de organizar y administrar recursos como procesador, memoria, 
dispositivos de E/S y almacenamiento.
