\chapter{ Justificación de la propuesta seleccionada}

\section{Nombre del sistema operativo base}
El nombre del sistema operativo es ``HelenOS''. El proyecto fue iniciado por Jakub Jermar en 2001, en ese momento era un código independiente que funcionaba como kernel para IA-32. \citep{decky2015application}

\section{Objetivos del proyecto original}
Según \citep{decky2015application}, este sistema operativo, tiene un enfoque educativo y experimental. Además, busca servir como una plataforma para la investigación y desarrollo de sistemas operativos de propósito general, teniendo muy en cuenta la fiabilidad y practicidad.
De manera más organizada se muestran los siguientes objetivos principales en forma de lista:
\begin{itemize}
    \item Proporcionar una plataforma real para el estudio y la experimentación en el diseño de sistemas operativos modernos.
    \item Implementar una arquitectura microkernel funcional que priorice la seguridad, la modularidad y la fiabilidad.
    \item Facilitar el aprendizaje de conceptos fundamentales de sistemas operativos mediante una implementación clara, accesible y extensible.
    \item Servir como base para el desarrollo incremental de servicios del sistema, controladores de dispositivos y mecanismos de comunicación entre procesos.
\end{itemize}

\section{Arquitectura y enfoque técnico}
El sistema operativo HelenOS se basó en la arquitectura multiserver de microkernel, donde el kernel de encarga de tareas muy específicas y principal; y los servicios son implementados en componentes aislados e independientes, en esta caso implementados con permisos de modo privilegiado para el kernel y modo usuario los demás \citep{decky2015application}.
A continuación, se incluye una vista general de la arquitectura/organización del kernel de HelenOS:
\begin{figure}[H]
\centering
\includegraphics[width=0.8\textwidth]{figures/overviewKernelArchitecturaHelenOS.jpg}
\centering
\caption{Vista general de la arquitectura/organización del kernel de HelenOS.}
\label{fig:overviewKernelArchitecturaHelenOS}
\centering
\small Fuente: Obtenido de \citep{decky2015application} \textit{Application of Software Components in Operating System Design}.
\end{figure}
El enfoque técnico es claro, al haber iniciado como un proyecto educativo independiente, se enfoca en la simplicidad y claridad del código, manteniendo la modularidad respectiva a su arquitectura elegida. En principio, HelenOS fue diseñado para la arquitectura IA-32, pero al día de hoy soporta múltiples arquitecturas, incluyendo x86-64, ARM y MIPS \citep{decky2015application}.

\section{Nivel de complejidad y adecuación al contexto educativo}
Considerando un entorno de estudio de sistemas operativos, justamente el entorno en el que se propicia esta
investigación, con conocimientos previos de C y fundamentos de sistemas operativos; es un proyecto con un 
nivel de complejidad media, más que nada por su tamaño (es extenso), además del uso de servicios mediante IPC
y también considerar que la arquitectura multiserver, que implica microkernel, no es la más sencilla tampoco.
Y eso sin contar la diversidad de arquitecturas soportadas, lo cual añade dificultad al análisis del código. 

Se escogió este sistema operativo debido a que es bastante adecuado para este contexto educativo, ya que su 
diseño modular permite su análisis en forma progresiva, empezando por componentes básicos y terminando en los 
más complejos. Además, otras de las razones para decir que es adecuado son su excelente estructura del código, 
la separación clarísima de sus componentes y, con más peso para este proyecto, su escalabilidad ya que podemos 
implementar de manera gradual más componentes de manera que se integre al sistema completamente.

\section{Comunidad, documentacion y soporte disponible}
Se consideró en la categorización del nivel de complejidad que, la accesibilidad para los estudiantes es
bastante buena, ya que el proyecto es open source y cuenta con una gran cantidad de estudios en torno a él.
Sin embargo, una consideración a tomar en cuenta es que, aunque HelenOS dispone de documentación formal 
(guías, artículos, documentación generada), esta puede no detallar cada componente del sistema operativo con 
la exhaustividad que otros sistemas operativos grandes ofrecen (por ejemplo un libro completo o wiki 
ultra-detallada), aunque esta deficiencia se ve compensada por la claridad del código y la estructura modular, 
sobre su comunidad es importante decir que no es muy grande, sin embargo es activamente estudiado a nivel académico 
y cuenta con foros donde se puede solicitar ayuda, en algunas redes sociales como reddit y GitHub.