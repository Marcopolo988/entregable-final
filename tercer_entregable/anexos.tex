\chapter*{Anexos}
\addcontentsline{toc}{chapter}{Anexos}

\section*{Anexo A: Comandos básicos de HelenOS}

La siguiente tabla presenta los comandos más utilizados en el shell Bdsh de HelenOS, que permiten la interacción básica con el sistema operativo:

\begin{longtable}{|p{3.5cm}|p{11cm}|}
\caption{Comandos básicos del shell Bdsh en HelenOS} \\
\hline
\textbf{Comando} & \textbf{Descripción} \\
\hline
\endfirsthead

\hline
\textbf{Comando} & \textbf{Descripción} \\
\hline
\endhead

\hline
\endfoot

\hline
\endlastfoot

\texttt{help} & Muestra la lista de comandos disponibles en el shell. \\
\hline
\texttt{ls [directorio]} & Lista el contenido del directorio especificado o del directorio actual. \\
\hline
\texttt{cd <directorio>} & Cambia al directorio especificado. \\
\hline
\texttt{pwd} & Muestra la ruta del directorio de trabajo actual. \\
\hline
\texttt{cat <archivo>} & Muestra el contenido de un archivo de texto. \\
\hline
\texttt{cp <origen> <destino>} & Copia un archivo del origen al destino. \\
\hline
\texttt{rm <archivo>} & Elimina el archivo especificado. \\
\hline
\texttt{mkdir <directorio>} & Crea un nuevo directorio. \\
\hline
\texttt{mount <tipo> <dispositivo> <punto>} & Monta un sistema de archivos. \\
\hline
\texttt{unmount <punto>} & Desmonta un sistema de archivos. \\
\hline
\texttt{tasks} & Muestra las tareas actualmente en ejecución. \\
\hline
\texttt{threads} & Muestra los hilos del sistema. \\
\hline
\texttt{kill <id>} & Termina la tarea con el identificador especificado. \\
\hline
\texttt{klog} & Muestra el registro del kernel. \\
\hline
\texttt{exit} & Finaliza la sesión del shell. \\
\hline

\end{longtable}

\section*{Anexo B: Estructura del código fuente de HelenOS}

El repositorio de HelenOS sigue una organización jerárquica que refleja su arquitectura modular. A continuación se describe la estructura principal de directorios:

\begin{longtable}{|p{4cm}|p{10.5cm}|}
\caption{Estructura de directorios del código fuente de HelenOS} \\
\hline
\textbf{Directorio} & \textbf{Contenido} \\
\hline
\endfirsthead

\hline
\textbf{Directorio} & \textbf{Contenido} \\
\hline
\endhead

\hline
\endfoot

\hline
\endlastfoot

\texttt{/kernel} & Código fuente del micronúcleo, incluyendo subsistemas de memoria, planificación, IPC y soporte de arquitecturas. \\
\hline
\texttt{/kernel/arch} & Código específico de cada arquitectura soportada (ia32, amd64, arm32, mips32, etc.). \\
\hline
\texttt{/kernel/generic} & Código genérico del kernel independiente de la arquitectura. \\
\hline
\texttt{/uspace} & Código de espacio de usuario, incluyendo bibliotecas, servidores y aplicaciones. \\
\hline
\texttt{/uspace/lib} & Bibliotecas de espacio de usuario (libc, libposix, libui, etc.). \\
\hline
\texttt{/uspace/srv} & Servidores del sistema (vfs, devman, locsrv, ns, etc.). \\
\hline
\texttt{/uspace/drv} & Controladores de dispositivos en espacio de usuario. \\
\hline
\texttt{/uspace/app} & Aplicaciones de usuario (bdsh, edit, tetris, etc.). \\
\hline
\texttt{/boot} & Archivos de arranque y configuración del bootloader. \\
\hline
\texttt{/tools} & Scripts y herramientas de desarrollo (toolchain.sh, ew.py, etc.). \\
\hline
\texttt{/abi} & Definiciones de la interfaz binaria de aplicaciones. \\
\hline

\end{longtable}

\section*{Anexo C: Proceso de compilación de HelenOS}

El proceso de compilación de HelenOS requiere seguir una serie de pasos específicos. A continuación se presenta la secuencia de comandos necesaria para compilar el sistema desde el código fuente:

\begin{verbatim}
# 1. Clonar el repositorio
git clone https://github.com/HelenOS/helenos.git
cd helenos

# 2. Instalar dependencias (Ubuntu/Debian)
sudo apt-get install build-essential wget texinfo flex bison \
    libgmp-dev libmpfr-dev libmpc-dev python3 python3-yaml \
    genisoimage xorriso

# 3. Compilar el toolchain cruzado (ejemplo para amd64)
cd tools
./toolchain.sh amd64
cd ..

# 4. Configurar la arquitectura objetivo
make PROFILE=amd64

# 5. Compilar HelenOS
make

# 6. Ejecutar en QEMU
tools/ew.py
\end{verbatim}

El script \texttt{toolchain.sh} descarga, configura y compila automáticamente las versiones compatibles de GCC y Binutils para la arquitectura seleccionada. Este proceso puede tomar varios minutos dependiendo de la capacidad del sistema.

\section*{Anexo D: Configuración de QEMU para HelenOS}

Para ejecutar HelenOS en el emulador QEMU, se pueden utilizar las siguientes opciones de configuración:

\begin{verbatim}
# Ejecución básica para arquitectura amd64
qemu-system-x86_64 -cdrom image.iso -m 256M -enable-kvm

# Ejecución con salida serial para depuración
qemu-system-x86_64 -cdrom image.iso -m 256M \
    -serial stdio -enable-kvm

# Ejecución con red habilitada
qemu-system-x86_64 -cdrom image.iso -m 256M \
    -netdev user,id=net0 -device e1000,netdev=net0

# Ejecución con múltiples CPUs (SMP)
qemu-system-x86_64 -cdrom image.iso -m 512M -smp 4
\end{verbatim}

El script \texttt{tools/ew.py} incluido en el repositorio de HelenOS automatiza la configuración de QEMU con los parámetros óptimos para cada arquitectura.

\section*{Anexo E: Diagrama de comunicación IPC}

El mecanismo de comunicación entre procesos (IPC) en HelenOS se basa en el modelo de paso de mensajes mediante ``teléfonos'' (phones) y ``answerboxes''. 
El flujo típico de una llamada IPC es el siguiente:
\begin{enumerate}
    \item El cliente obtiene un teléfono (phone) conectado al servidor destino.
    \item El cliente envía un mensaje a través del teléfono usando \texttt{ipc\_call\_sync()} o \texttt{ipc\_call\_async()}.
    \item El kernel copia el mensaje corto (hasta 5 argumentos de 64 bits) al answerbox del servidor.
    \item El servidor recibe el mensaje mediante \texttt{ipc\_wait\_for\_call()}.
    \item El servidor procesa la solicitud y prepara la respuesta.
    \item El servidor envía la respuesta usando \texttt{ipc\_answer()}.
    \item El kernel entrega la respuesta al cliente, desbloqueándolo si la llamada era síncrona.
\end{enumerate}

El siguiente diagrama ejemplifica este flujo de comunicación:
\begin{figure}[H]
	\centering
	\includegraphics[width=0.9\textwidth]{figures/diagramatecnicoE.png}
	\caption{Diagrama de comunicación IPC en HelenOS (fuente: \citep{helenos_wiki_tutorial}).}
	\label{fig:Comunicación IPC}
\end{figure}
Para transferencias de datos mayores, HelenOS utiliza áreas de memoria compartida que se establecen mediante llamadas específicas del IPC, evitando así la copia excesiva de datos entre espacios de direcciones.

\section*{Anexo F: Glosario de términos}

\begin{longtable}{|p{4cm}|p{10.5cm}|}
\caption{Glosario de términos técnicos} \\
\hline
\textbf{Término} & \textbf{Definición} \\
\hline
\endfirsthead

\hline
\textbf{Término} & \textbf{Definición} \\
\hline
\endhead

\hline
\endfoot

\hline
\endlastfoot

\textbf{Microkernel} & Arquitectura de sistema operativo donde el núcleo implementa solo funciones esenciales (IPC, planificación básica, gestión de memoria), delegando servicios a procesos de usuario. \\
\hline
\textbf{Multiservidor} & Diseño donde los servicios del sistema operativo se ejecutan como servidores independientes en espacio de usuario. \\
\hline
\textbf{IPC} & Inter-Process Communication. Mecanismo que permite la comunicación entre procesos aislados. \\
\hline
\textbf{VFS} & Virtual File System. Capa de abstracción que unifica el acceso a diferentes sistemas de archivos. \\
\hline
\textbf{Tarea (Task)} & Unidad de recursos en HelenOS que agrupa hilos y define un espacio de direcciones. \\
\hline
\textbf{Hilo (Thread)} & Unidad básica de ejecución en el kernel de HelenOS. \\
\hline
\textbf{Fibrilla (Fibril)} & Hilo cooperativo en espacio de usuario, implementado sobre la API del kernel. \\
\hline
\textbf{SLAB allocator} & Algoritmo de asignación de memoria optimizado para objetos pequeños y frecuentes. \\
\hline
\textbf{Buddy system} & Algoritmo de asignación de memoria física que maneja bloques de tamaño potencia de dos. \\
\hline
\textbf{TLB} & Translation Lookaside Buffer. Caché de traducciones de direcciones virtuales a físicas. \\
\hline
\textbf{SMP} & Symmetric Multiprocessing. Arquitectura con múltiples procesadores compartiendo memoria. \\
\hline
\textbf{Teléfono (Phone)} & Abstracción de HelenOS para representar un canal de comunicación IPC hacia un servidor. \\
\hline
\textbf{Cross-compiler} & Compilador que genera código para una plataforma diferente a la que se ejecuta. \\
\hline
\textbf{Bdsh} & Brain-dead shell. Shell de línea de comandos por defecto en HelenOS. \\
\hline

\end{longtable}

\section*{Anexo G: Recursos adicionales}

Para profundizar en el estudio de HelenOS, se recomiendan los siguientes recursos:

\begin{itemize}
    \item \textbf{Sitio web oficial:} \url{https://www.helenos.org/} -- Contiene documentación, noticias y enlaces a recursos del proyecto.
    
    \item \textbf{Repositorio GitHub:} \url{https://github.com/HelenOS/helenos} -- Código fuente completo del sistema operativo.
    
    \item \textbf{Wiki del proyecto:} \url{https://www.helenos.org/wiki/} -- Tutoriales, guías de desarrollo y documentación técnica.
    
    \item \textbf{Documentación de diseño:} \url{https://www.helenos.org/doc/design.pdf} -- Documento técnico sobre la arquitectura y diseño del sistema.
    
    \item \textbf{Tesis relacionadas:} El repositorio de la Universidad Charles de Praga contiene múltiples tesis de grado y maestría sobre diferentes aspectos de HelenOS.
    
    \item \textbf{Canal IRC:} \texttt{\#helenos} en \texttt{irc.libera.chat} -- Canal de comunicación con la comunidad de desarrolladores.
\end{itemize}

\section*{Anexo H: Guía de operación paso a paso}

Este manual detalla el flujo de interacción con HelenOS 0.14.1 (Aladar) desde el encendido hasta el apagado del sistema.

\subsection*{Paso 1: Selección en el gestor de arranque}
Al iniciar la máquina virtual en QEMU, la primera pantalla que aparece es el menú de GNU GRUB. El usuario debe utilizar las teclas de dirección para resaltar la opción \texttt{HelenOS 0.14.1} y presionar la tecla \textit{Enter} para iniciar el proceso de carga del núcleo.

\begin{figure}[H]
    \centering
    \includegraphics[width=1\textwidth]{figures/01_grub_menu.jpeg}
    \caption{Menú de inicio de GRUB para la selección del sistema operativo.}
\end{figure}

\subsection*{Paso 2: Carga de módulos del sistema}
Una vez seleccionado el sistema, se muestra una pantalla de carga de archivos en modo texto. En este punto, el cargador transfiere a la memoria RAM el kernel, los servidores básicos (como el sistema de archivos virtual y el servidor de nombres) y la imagen del disco RAM inicial (\texttt{initrd.img}).

\begin{figure}[H]
    \centering
    \includegraphics[width=1\textwidth]{figures/02_loading_modules.jpeg}
    \caption{Proceso de carga de módulos y componentes esenciales en memoria.}
\end{figure}

\subsection*{Paso 3: Inicialización y log de servicios}
Posteriormente, el sistema comienza a levantar los servicios de hardware y red. El usuario podrá observar en tiempo real cómo se configuran las conexiones de red (DHCP), los dispositivos de entrada (teclado y ratón) y el servidor de video, finalizando con el inicio del servicio de sonido \texttt{hound} y el servicio de pantalla.

\begin{figure}[H]
    \centering
    \includegraphics[width=1\textwidth]{figures/03_service_log.jpeg}
    \caption{Registro de inicialización de servicios de red y hardware.}
\end{figure}

\subsection*{Paso 4: Ingreso al entorno de escritorio}
Tras la carga de servicios, aparece el escritorio gráfico azul. Por defecto, se abre una ventana de Terminal que da la bienvenida al usuario con información sobre la arquitectura (amd64) y la revisión del sistema. En esta ventana se puede escribir el comando \texttt{help} para ver consejos de uso.

\begin{figure}[H]
    \centering
    \includegraphics[width=1\textwidth]{figures/04_desktop_main.jpeg}
    \caption{Pantalla de bienvenida y terminal inicial en el escritorio.}
\end{figure}

\subsection*{Paso 5: Monitoreo y herramientas de cálculo}
El usuario puede interactuar con múltiples aplicaciones simultáneamente. En la captura se observa el uso del comando \texttt{top} en la terminal para monitorear el consumo de recursos (CPU y memoria) y el estado de las tareas. Al mismo tiempo, se puede utilizar la \textit{Calculator} para realizar operaciones matemáticas básicas.

\begin{figure}[H]
    \centering
    \includegraphics[width=1\textwidth]{figures/05_top_calculator.jpeg}
    \caption{Uso simultáneo del monitor de sistema (top) y la calculadora gráfica.}
\end{figure}

\subsection*{Paso 6: Entretenimiento y juegos}
HelenOS incluye aplicaciones de ocio como el juego Tetris, ejecutable dentro de la terminal. Para jugar, el usuario utiliza las teclas \texttt{j} (izquierda), \texttt{l} (derecha) y \texttt{k} (rotar). El sistema permite mantener el monitor de procesos activo en segundo plano mientras se juega.

\begin{figure}[H]
    \centering
    \includegraphics[width=1\textwidth]{figures/06_tetris_game.jpeg}
    \caption{Ejecución del juego Tetris en el entorno de consola.}
\end{figure}

\subsection*{Paso 7: Gestión de ventanas y Barber Pole}
La barra de tareas inferior permite alternar entre las diferentes ventanas abiertas. Además de las herramientas de texto, se puede ejecutar la aplicación \textit{Barber Pole}, que muestra una animación gráfica constante. La interfaz soporta el cambio de foco y la organización de múltiples utilidades en pantalla.

\begin{figure}[H]
    \centering
    \includegraphics[width=1\textwidth]{figures/07_barber_pole.jpeg}
    \caption{Interfaz con múltiples tareas activas y demostración gráfica.}
\end{figure}

\subsection*{Paso 8: Apagado del sistema}
Para finalizar la sesión, el usuario debe hacer clic en el botón \textit{Start} y seleccionar la opción de salida. Se desplegará una ventana de confirmación denominada \textit{Shutdown}, donde se puede elegir entre \textit{Power off} para apagar completamente la máquina virtual o \textit{Restart} para reiniciar el sistema operativo.

\begin{figure}[H]
    \centering
    \includegraphics[width=1\textwidth]{figures/08_shutdown_dialog.jpeg}
    \caption{Menú de confirmación para el apagado seguro o reinicio.}
\end{figure}