\chapter{Argumentos pedagógicos}

\section{Claridad conceptual y didáctica}
HelenOS tiene una arquitectura bien clara y definida, con su organización en componentes y módulos favorece la compración de los conceptos fundamentales de los sistemas operativos. HelenOS muestra de manera directa mecanismos para mantener la modularidad como la comunicación entre procesos (IPC), planificación de hilos y gestión de memoria; facilitando y clarificando aún más el análisis conceptual del sistema operativo en profundidad. En conclusión es un sistema operativo muy didáctico y con mucho potencial para ser usado en la enseñanza de los cursos de sistemas operativos.

\section{Potencial para fomentar el aprendizaje activo}
El uso de este sistema operativo como base del proyecto apoya y contribuye al aprendizaje activo, ya que los estudiantes participaremos directamente en la contrucción, modificación, agregación y prueba de un sistema operativo real. El proceso de desarrollo implica proactividad, para aprender a configurar y compilar el sistema, implica también análisis del flujo de ejecución y componentes individuales del sistema operativo. Además, al implementar nuevas funcionalidades por cuenta propia, se puede llegar a cometer errores cuya solución refuerza el conocimiento adquirido a partir de este aprendizaje experimental. En específico, este sistema operativo tiene un montón de potencial para fomentar el aprendizaje activo sobre todo si se quiere enseñar algo más moderno y aplicable actualmente, a diferencia de opciones más sencillas como Minix o XV6 al contar con arquitecturas muy sencillas o limitadas, así como también diferente a opciones más complejas que pueden llevar a una parálisis por sobrecarga de información, como Theseus o Linux. Razones por las cuales se escogió HelenOS al considerarlo un equilibro adecuado.

\section{Relación con los contenidos del curso}
El sistema operativo HelenOS está muy bien alineado a los contenidos del curso específico de la investigación, incluyendo en su código todos los temas avanzados (como planificación de procesos e hilos, gestión de memoria, comunicación entre procesos (IPC), sistemas de archivos y la gestión de E/S) varios de ellos con incluso más profundidad de la que nos permitió el avance del curso; así como también conceptos relativos a sistemas operativos modernos, tales como la gestión de drivers como un árbol y soporte para múltiples tipos de buses. Se asegura un correspondencia casi perfecta entre la teoría impartida en el curso y su implementación práctica en HelenOS, permitiendo así que los estudiantes relacionemos los conceptos abstractos con soluciones concretas.

\section{Posibilidad de trabajo colaborativo y evaluación progresiva}
En HelenOS y gracias a su estructura modular, es fácilmente aplicable un trabajo en equipo y una planificación en varias etapas. En cada etapa, cada componente independiente puede ser asignado a uno de los integrantes del equipo; queremos resaltar que identificamos que el proyecto era altamente compatible para aplicar la metodología ágil Scrum, aplicando sus principios, asegurando así un trabajo en equipo efectivo, una evaluación de calidad constante y progresiva, y en general un buen desarrollo del proyecto; se aplicó de forma que las etapas identificadas en el cronograma fueron tratadas como sprints, y se aseguró calidad al entregarse incrementos de documentación e implementación al final de cada sprint \citep{SBOKGuide4th_Spanish}. En esencia, HelenOS posee una altísima posibilidad de trabajo en equipo y evaluaión progresiva debido en gran medida a su arquitectura microkernel multiserver y su excelente organización modular.