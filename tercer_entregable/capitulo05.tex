\chapter{Herramientas y entorno de desarrollo}

HelenOS es un sistema operativo académico multiescala (``microkernel + servidores de usuario'') desarrollado en C. Prácticamente todo el código nativo del sistema (núcleo y servidores) está escrito en C, que es el único lenguaje con un \textit{runtime} nativo completo. 
Algunos componentes muy bajo nivel (por ejemplo, el arranque o manejadores de interrupción) se implementan en ensamblador específico de cada CPU para manejar detalles de hardware. según \citep{jindrak2022cpp} pag. 3.

\section{Lenguaje(s) de programación alto y bajo nivel}
No existe aún un soporte nativo oficial para C++ o lenguajes de alto nivel: actualmente C++ y Python sólo pueden usarse vía puertos experimentales \citep{jindrak2022cpp}.

Como indica la documentación, HelenOS ``puede ejecutar casi exclusivamente programas en C... tiene soporte para C++ y Python, pero con limitaciones significativas, por lo que estos lenguajes no se usan ampliamente''. 
Adicionalmente, se han emprendido proyectos de investigación para portar lenguajes modernos; por ejemplo, recientes trabajos buscan habilitar Rust en HelenOS \citep{volf2025rust}.

Pordemos observar Interfaz gráfica de HelenOS (versión 0.11.2) la Figura \ref{fig:interfaz_usuario}.
\begin{figure}[H]
	\centering
	\includegraphics[width=0.9\textwidth]{figures/interfaz_usuario.png}
	\caption{Interfaz de usuario. (fuente: \citep{helenos_wiki_tutorial}).}
	\label{fig:interfaz_usuario}
\end{figure}

HelenOS dispone de línea de comandos y un GUI propio para usuario. Como resume la documentación oficial: ``Tenemos línea de comandos y una interfaz gráfica sencilla que permiten manipular archivos, ejecutar aplicaciones y montar sistemas de ficheros... se 
puede jugar al Tetris o editar archivos de texto... También tenemos redes y se ha portado software de desarrollo (GCC, binutils, Python, pcc)'' \citep{helenos_wiki_tutorial}. En la figura se ve una sesión típica en HelenOS con varias aplicaciones gráficas. En suma, el entorno de usuario 
de HelenOS incluye utilidades básicas (editor de texto, gestor de archivos, demos gráficas, etc.) soportadas directamente por el sistema, pero el núcleo del OS y sus servidores están escritos casi exclusivamente en C (con ayuda puntual de ensamblador).

\section{Compilador cruzado}
Para compilar HelenOS se requiere un \textit{toolchain} cruzado específico. El proyecto provee un script (\texttt{tools/toolchain.sh}) que genera un compilador cross-GCC para cada arquitectura objetivo (amd64, ia32, arm, etc.) (\citep{decky2015application}). 
No se puede usar el compilador nativo del host para construir HelenOS: la documentación advierte claramente que el compilador del sistema 
produce binarios incompatibles y que sólo se ha probado HelenOS con la versión de GCC instalada por el script (\citep{decky2010road}). El compilador por defecto de HelenOS es GCC ; 
existe parche \texttt{-helenos-} para hacer el compilador ``nativo'' de HelenOS. Alternativamente, HelenOS puede construirse con Clang en arquitecturas comunes (ia32, amd64), aunque Clang no cubre todas las plataformas soportadas. 
el flujo de desarrollo típico es usar GCC (o Clang en X86) como \textit{cross-compiler}, generado por el script oficial, para compilar bibliotecas y aplicaciones HelenOS.

\section{Emulador }
Las pruebas y demostraciones de HelenOS se realizan sobre todo en máquinas virtuales/emuladores. 
El Emulador recomendado y más usado es QEMU: existe una página dedicada (``Running HelenOS in QEMU'') que explica cómo lanzar imágenes de HelenOS en QEMU. 
El proyecto incluye un script de envoltura (\texttt{tools/ew.py}) que inicia automáticamente QEMU con las opciones adecuadas para la configuración elegida. 
La documentación de HelenOS menciona explícitamente ``RunningInVirtualBox''con instrucciones para VirtualBox, 
y en el repositorio se proporciona incluso un ejemplo de configuración de VMware (archivo \texttt{.vmx}) para arrancar HelenOS en VMware Workstation. \citep{helenos_wiki_tutorial}.

\begin{figure}[H]
	\centering
	\includegraphics[width=0.9\textwidth]{figures/qemu.png}
	\caption{QEMU es un emulador y virtualizador de código abierto capaz de realizar emulación de sistema completo en múltiples arquitecturas (fuente: \citep{helenos_wiki_tutorial}).}
	\label{fig:qemu}
\end{figure}

\section{Control de versiones (Git)}
HelenOS gestiona su código fuente con Git. Los repositorios oficiales están alojados en GitHub: el repositorio principal (\citep{helenos_wiki_tutorial}) contiene el OS base, y existen repositorios adicionales (``harbours'' para software portado y ``ci'' para integración continua). 
En sus FAQ oficiales se indica que ``el código fuente de HelenOS actualmente se gestiona con Git... los desarrolladores son alentados a alojar sus ramas en GitHub (github.com/HelenOS)''. Antes se usaba Bazaar, pero hoy todo se ha migrado a Git. Como buenas prácticas, el equipo mantiene ramas bien 
identificadas (por versión/milestone) y usa revisiones por \textit{pull requests} en GitHub.

\section{Editor o entorno de desarrollo}
La documentación oficial de HelenOS no prescribe un editor de código o IDE específico para desarrollar el sistema. En la práctica, los desarrolladores utilizan los editores y entornos comunes de la comunidad (C/C++) que prefieran. Muchos usan editores de texto avanzados (por ejemplo, \textbf{Vim o Emacs})
o entornos gráficos como \textbf{Visual Studio Code} con extensiones de C/C++, dado que HelenOS es esencialmente software en C. HelenOS incluye también un editor básico propio (comando \texttt{edit} en la consola), pero para escribir el código fuente generalmente se prefiere un editor externo 
con resaltado de sintaxis y soporte de \textit{debugging}. En ausencia de una recomendación oficial, cada desarrollador escoge su entorno de desarrollo habitual (p.ej. VS Code, CLion, Vim/Neovim, etc.) para editar el código de HelenOS.