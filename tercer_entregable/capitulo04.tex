\chapter{Diseño del sistema operativo propuesto}

Este capítulo presenta el diseño del sistema operativo HelenOS, Diseñado a partir de una arquitectura de micronúcleo y entornos operativos
multiservidor \citep{decky2015application,decky2010road,helenos_GitHub}. Se detallan el diseño estructural y los módulos esenciales para el desarrollo, 
las políticas de planificación y manejo de recursos y el flujo de ejecución esperado desde el arranque hasta la interacción del usuario.
Se implementa un núcleo básico apoyado por múltiples servidores que operan en el área de usuario como redes, archivos y controladores, comunicándose
mediante paso de mensajes asíncrono \citep{decky2015application}. Ademas se inspirado en Unix/POSIX, evita interfaces heredadas cuando 
existen alternativas modernas, como por ejemplo, prescinde de los sockets de POSIX y expone una API orientada a flujos TCP \citep{korop2025packet,helenos_project}.
La ruta de la comunicación de red atraviesa por procesos (NIC \textrightarrow{} Ethernet \textrightarrow{} IP \textrightarrow{} TCP) 
antes de llegar a la aplicación, reforzando el aislamiento y la modularidad \citep{korop2025packet}.
Finalmente, los ejecutables siguen el formato ELF con soporte de enlace dinámico, PIE y TLS, preparados por un servidor de carga antes del inicio de cada tarea \citep{volf2025rust}.

\section{Diagrama de arquitectura general}
El sistema operativo HelenOS está diseñado como un microkernel relativamente pequeño, asistido por un conjunto de controladores de espacio
de usuario y tareas de servidor. HelenOS no es muy radical en cuanto a qué subsistemas deben o no implementarse en el kernel, en algunos
casos, existen tanto controladores de kernel como de espacio de usuario. La razón para crear el sistema como un microkernel es prosaica.
Si bien inicialmente es más difícil obtener el mismo nivel de funcionalidad de un microkernel que en el caso de un kernel monolítico simple,
un microkernel es mucho más fácil de mantener una vez que sus componentes se han puesto en funcionamiento. Por lo tanto, el kernel de HelenOS,
así como sus bibliotecas esenciales de espacio de usuario, solo pueden ser mantenidas por unos pocos desarrolladores que las comprendan completamente.
Además, un sistema operativo basado en microkernel se completa antes que los kernels monolíticos, ya que el sistema puede utilizarse
incluso sin algunos subsistemas tradicionales (por ejemplo, dispositivos de bloque, sistemas de archivos y redes). Segun \citep{helenos_design_2006}.

Pordemos observar el diagrama de arquitectura general en la Figura \ref{fig:Arquitectura HelenOS}.
\begin{figure}[H]
	\centering
	\includegraphics[width=0.9\textwidth]{figures/Arquitectura HelenOS.png}
	\caption{Diseño general de HelenOS y organización multiservidor basada en microkernel (fuente: \citep{helenos_design_2006}).}
	\label{fig:Arquitectura HelenOS}
\end{figure}

\section{Componentes a implementar}

\subsection{Bootloader}
En la arquitectura x86-64, HelenOS emplea el gestor de arranque GRUB (Grub boot loader). Segun \citep{helenos_wiki_tutorial} para iniciar el sistema. Este se encarga de cargar el kernel junto
con un conjunto inicial de tareas de espacio de usuario necesarias para completar el proceso de arranque. Asimismo, GRUB carga un disco
RAM que contiene el sistema de archivos raíz. Durante las fases iniciales del arranque, el sistema muestra mensajes de registro generados
tanto por el kernel como por las tareas de espacio de usuario a medida que se inicializan. Una vez completado este proceso, el compositor
de la interfaz gráfica toma control de la pantalla, momento en el cual el sistema ya cuenta con más de 35 tareas de espacio de usuario en
ejecución, responsables de proporcionar la funcionalidad básica del sistema.
Según la documentación de diseño y del propio proyectos \citep{helenos_design_2006,helenos_prjdoc_2006}.

\subsection{Kernel básico}
Segun \citep{decky2010road}El planificador de HelenOS mantiene varias colas de ejecución asociadas a cada procesador. Los hilos preparados para ejecutarse se insertan en estas colas de acuerdo con su nivel
de prioridad y el procesador actual, desde donde son seleccionados para su ejecución. Para garantizar un reparto equilibrado de la carga, existen hilos del núcleo con funciones
especiales encargados de migrar hilos entre procesadores cuando es necesario. La planificación se basa en una política de round-robin aplicada sobre múltiples colas de prioridad.

Aunque el diseño del micronúcleo prioriza la simplicidad conceptual, HelenOS hace uso de mecanismos modernos y eficientes, incluyendo árboles AVL y B+, tablas hash, un asignador
de memoria SLAB, diversas primitivas de sincronización interna y técnicas de bloqueo de grano fino, con el objetivo de mejorar el rendimiento y la escalabilidad del sistema.

\subsection{Gestión de procesos}
En HelenOS, el hilo es la unidad básica de ejecución del kernel y se agrupa en tareas según su funcionalidad. En el espacio de usuario se emplean fibrillas, hilos cooperativos
construidos sobre una API del núcleo y utilizados por el marco asíncrono. Debido a su arquitectura de micronúcleo, la comunicación entre procesos (IPC) es fundamental.
Las tareas intercambian información mediante mensajes breves o mediante el compartimiento de memoria para datos de mayor tamaño. El modelo de IPC permite múltiples conexiones
simultáneas entre tareas. \citep{helenos_design_2006,jindrak2022cpp}.

\subsection{Gestión de memoria}
HelenOS garantiza la coherencia entre la TLB y las tablas de páginas mediante un mecanismo de invalidación coordinada en sistemas multiprocesador. La gestión de memoria del
sistema cubre la asignación para el kernel, la traducción de direcciones virtuales y la administración de espacios de direcciones. Cada espacio de direcciones está compuesto
por áreas disjuntas respaldadas por memoria anónima, imágenes ejecutables o memoria física. El sistema permite compartir áreas entre tareas, pero no soporta intercambio de
páginas con almacenamiento secundario. \citep{helenos_design_2006}

\subsection{Sistema de archivos}
El sistema operativo HelenOS, tiene el soporte de sistemas de archivos se basa en un Sistema de Archivos Virtual (VFS), que actúa como una capa de abstracción entre las
aplicaciones y los distintos sistemas de archivos. El VFS se implementa como un servidor central encargado de unificar el acceso a los dispositivos de almacenamiento.
Cada sistema de archivos se ejecuta como un servicio independiente en espacio de usuario y registra sus capacidades en el VFS. El VFS proporciona una interfaz común de
operaciones y ofrece compatibilidad con POSIX mediante una capa de adaptación. Su diseño se divide en un frontend, que gestiona solicitudes simples, y un backend, que
delega las operaciones al servidor de archivos correspondiente. \citep{zarevucky2012improved,cimerman2025raid}.
\subsection{Interfaz de usuario}
HelenOS proporciona una interfaz de usuario básica basada en texto, sin soporte gráfico integrado en el núcleo. La interacción con el sistema se realiza mediante un shell en
espacio de usuario, denominado Bdsh, encargado de interpretar comandos, ejecutar aplicaciones y gestionar la entrada y salida estándar. Dado su enfoque como sistema operativo
de investigación, el entorno de usuario es minimalista y se limita a una interfaz de línea de comandos (CLI). Las aplicaciones se comunican con los servidores del sistema a 
través del API de HelenOS, actuando el shell como un cliente de dichos servicios sin requerir soporte especial del kernel. Segun \citep{helenos_wiki_tutorial}.

\begin{figure}[H]
	\centering
	\includegraphics[width=0.9\textwidth]{figures/Diseño de HelenOS.png}
	\caption{Diseño general de HelenOS y organización multiservidor basada en microkernel (fuente: \citep{helenos_wiki_tutorial}).}
	\label{fig:Interfaz HelenOS}
\end{figure}
\subsection{Políticas de planificación y manejo de recursos}
HelenOS emplea un planificador preventivo con retroalimentación de prioridad, compatible con sistemas SMP y diseñado para ser altamente
portable. Actualmente soporta múltiples arquitecturas de hardware, incluyendo x86, x86-64, IA64, SPARC, PowerPC, ARM y MIPS. Aunque no 
está orientado al uso general debido a la falta de aplicaciones de usuario, ya cuenta con subsistemas esenciales como sistemas de archivos
y redes TCP/IP. El planificador utiliza múltiples colas de ejecución por procesador y una política round-robin con prioridades, incorporando
migración de hilos para balancear la carga. El diseño del sistema sigue el principio de separación entre mecanismos y políticas, delegando
estas últimas al espacio de usuario. \citep{decky2010road,helenos_design_2006}.


\subsection{Flujo de ejecución básico}
El flujo de ejecución de HelenOS inicia con el bootloader, que carga el microkernel en memoria y transfiere el control. El kernel
inicializa el hardware esencial y habilita la planificación, la gestión de memoria y el IPC. Luego se lanzan los servidores fundamentales
en espacio de usuario, como controladores de dispositivos, VFS y servicios de nombres. Finalmente, se inicia el shell Bdsh, y el sistema
opera mediante múltiples tareas independientes que se comunican de forma concurrente a través del IPC del microkernel.\citep{helenos_design_2006}
