\section{Haiku OS}

\subsection{Nombre del proyecto o sistema operativo}
Según \citep{leavengood2012haiku}, Haiku Os es un proyecto visionario con el nombre del sistema operativo Haiku OS en la actualidad, pero que antes era OpenBeOS, un sistema operativo fundado en la División de Corporaciones del Estado de Nueva York como una organización sin fines de lucro en julio del 2003. por su fundador Michael Phipps.
El nombre del sistema operativo proviene de la palabra japoness ``Haiku``, por que era usada en los mensajes de error de BeOS, eran mensajes presentados de forma poética japonesa.

\subsection{Enlace al repositorio y/o documentación oficial}
\begin{itemize}
    \item \textbf{Repositorio oficial (espejo en GitHub):} \url{https://github.com/haiku/haiku} — funciona como un \textit{mirror} del repositorio principal alojado en el servidor de Haiku, donde se gestiona el desarrollo activo del sistema operativo. (Haiku Project)
    \item \textbf{Repositorio principal (Gerrit):} \url{https://review.haiku-os.org/haiku} — servidor oficial utilizado para la revisión y aprobación del código fuente. (Haiku Project)
    \item \textbf{Sitio web oficial del proyecto:} \url{https://www.haiku-os.org/get-haiku/installation-guide/}
\end{itemize}

\subsection{Objetivo del proyecto}
Según \citep{stimac2011haiku} Haiku es un proyecto comunitario orientado al escritorio experimental, como meta ofrecer un sistema operativo de escritorio abierto, ligero y fácil de usar, diseñado especificamente para la computación personal inspirado en BeOS, Haiku se creo con la finalidad de recrear experiencia de uso para entender el funcionamiento y realizar mejoras. 

\subsection{Lenguaje(s) de implementación}
Haiku esta implementado en C++, haciendo uso de una API de Be orientada a objetos (The kits) que facilita el desarrollo de aplicaciones y controladores de dispositivos \citep{leavengood2012haiku}.
incluye contenedores de la Biblioteca de Plantillas Estandar de C++ (STL) y otras bibliotecas de terceros como OpenSSL, SQLite y zlib \citep{haiku_website_2025}.

\subsection{Arquitectura del sistema (monolítica, microkernel, etc.)}
El sistema operativo esta basado en BeOS un Kernel hibrido modular. Su núcleo se desarrollo de una rama de NewOS, la modularidad permite a los controladores y otros componentes que se carguen de manera dinamica todo lo proceso segun sea necesario. 
La siguientes son las capas principales que conforman el sistema operativo, según \citep{2025deployment}:
\begin{itemize}
    \item  \textbf{Capa de nucleo: } El nucleo de NewOS sirve para servicio como: multitarea, gestión de memoria, controladores y administrador de archivos. 
    \item \textbf{Servidor de aplicaciones: } El servido de aplicaciones gestiona las ventanas, el dibujo y el renderizado. Esta diseñado para un alto rendimiento y se integra con el servidor de entrada y el kit de herramientas.
    \item \textbf{Gestión de paquetes: } Haiku es compatible con Be File System (BFC) que incluye atributos de archivos extendidos.
    \item \textbf{Controlador de red:} Haiku incluye una pila de red modular compatible con estaciones.
\end{itemize}
En la siguiente imagen es al arquitectura del sistema operativo BeOS (el precursor de HaikuOS). Se puede observar que entre el hardware y las aplicaciones del software se encuentra el software BeOS. El software del sistema operativo tiene 3 capas, capa de micronucleo, capa de servidor y capa de software.
\begin{figure}[H]
\centering
\includegraphics[width=1\textwidth]{figures/ArquitecturaBeos.png}
\centering
\caption{Arquitectura del sistema operativo BeOS.}
\label{fig:arquitecturaBeOS}
\centering
\small Fuente: Obtenido de \citep{beinc_beos_bible_kits} 
\end{figure}

\subsection{Componentes implementados (procesos, memoria, archivos, etc.)}
En HaikuOS, como es característico en sistemas operativos de arquitectura híbrida modular, los componentes principales están implementados y se ejecutan en modo kernel, pero con una alta modularidad organizada en “kits” y servidores.
A continuación se presenta una tabla con los principales componentes y subsistemas de HaikuOS, su ubicación en el repositorio oficial y la idea técnica detrás de su implementación.
\begin{table}[H]
\centering
\caption{Funcionamiento técnico (idea clave) de los principales subsistemas de HaikuOS}
\begin{tabular}{|p{3.5cm}|p{5cm}|p{6cm}|}
\hline
\textbf{Componente / Sub-sistema} & \textbf{Nombre / Ruta en repositorio} & \textbf{Funcionamiento técnico (idea clave)} \\ \hline
Gestión de memoria & \texttt{src/system/kernel/vm/}, \texttt{paging.cpp} & Asignar páginas, espacio de direcciones por proceso. \\ \hline
Scheduling / multitarea & \texttt{src/system/kernel/ thread.cpp}, \texttt{scheduler.cpp} & Soporte para multiprocesador, planificación preventiva y cambio de contexto eficiente. \\ \hline
Sistema de archivos & \texttt{src/add-ons/kernel/ file\_systems/} & BFS extendido, con journalling; soporte FAT para compatibilidad. \\ \hline
Drivers / E/S & \texttt{src/add-ons/kernel/ {drivers, bus\_managers}/} & Implementados como módulos independientes y jerarquizados. \\ \hline
Servicios de red & \texttt{src/add-ons/kernel/ network/}, \texttt{src/servers/net\_server/} & Pila TCP/IP modular, soporte para sockets BSD, DHCP y servicios de red como servidores. \\ \hline
Interfaz gráfica / servicios de usuario & \texttt{src/servers/app/}, \texttt{src/servers/input/}, \texttt{src/kits/interface/}, \texttt{app\_server}, \texttt{input\_server} & Servidores para gestión gráfica y entrada, comunicación mediante kits orientados a objetos (GUI toolkit). \\ \hline
\end{tabular}
\centering
Fuente: Elaboración propia con base en \citep{haiku_GitHub,2025deployment}.
\end{table}

\subsubsection*{Gestión de memoria}
En HaikuOS se gestiona la memoria virtual con asignación de páginas bajo demanda, con método indexado y vector de bits para administrar espacio libre; y cada proceso tiene su propio espacio de direcciones. El módulo de “Virtual Memory Manager” se encarga de la asignación física y lógica de memoria \citep{Gonzales2017haiku}.

\subsubsection*{Scheduling / multitarea}
La planificación de hilos en el sistema operativo HaikuOS soporta más de un núcleo. Cada hilo tiene su propio contexto; el scheduler organiza prioridades (para usar el CPU) y cambios de contexto \citep{2025deployment}.

\subsubsection*{Sistema de archivos}
En HaikuOS, se tiene el sistema de archivos BFS (Be File System) con metadatos extendidos (pares clave-valor) y búsquedas indexadas. Además tiene soporte para otros sistemas como FAT para compatibilidad \citep{2025deployment,haiku_GitHub}.

\subsubsection*{Drivers / E/S}
Los drivers están organizados en forma de árbol con E/S (la cual está muy optimizada para contenido multimedia) en forma de módulos independientes, para permitir actualizarse en ejecución \citep{haiku_website_2025,beinc_beos_bible_kits}.

\subsubsection*{Servicios de red}
El sistema operativo HaikuOS, cuenta con un kit (Network Kit) muy completo y el servidor \texttt{net\_server} para comunicaciones. Tiene soporte para los protocolos TCP/IP y UDP, incluso, si le añades complementos, AppleTalk o IPX \citep{2025deployment,beinc_beos_bible_kits}.

\subsubsection*{Interfaz gráfica / servicios de usuario}
La interfaz gráfica de usuario de Haiku es gestionada por los servidores \texttt{app\_server} e \texttt{input\_server}. Los “kits” de desarrollo (Application, Interface) nos proporcionan APIs con el paradigma orientado a objetos en C++ para aplicaciones y servicios \citep{2025deployment}.

\subsection{Herramientas utilizadas (compiladores, emuladores, etc.)}
HaikuOS utiliza diversas herramientas para ejecutar, compilar y construir. También usa máquinas virtuales e implementaciones con funcionalidades adicionales. Entre las principales herramientas se encuentran:
\begin{itemize}
    \item \textbf{Compilador:} Uso de GCC 2 (modificado), para mantener compatibilidad con BeOS, y GCC 13 para el desarrollo actual \citep{Installing2024haiku}.
    \item \textbf{Emuladores y máquinas virtuales:} En \citep{Installing2024haiku} se hace uso de VirtualBox y en su página oficial \citep{haikuos2025} se recomienda usar QEMU, así que HaikuOS puede ejecutarse en ambos entornos.
    \item \textbf{Sistema de construcción:} Haiku modifica la herramienta “Jam” para compilar el kernel, los drivers y la interfaz gráfica. También ofrece compilación cruzada para otras arquitecturas \citep{haikuos2025}.
    \item \textbf{Herramientas de portabilidad:} En el repositorio \citep{haiku_GitHub}, se encuentran scripts \textit{raw disk}, \textit{MMC} o \textit{SD card} para generar imágenes en diferentes arquitecturas.
\end{itemize}

\subsection{Nivel de complejidad y accesibilidad para estudiantes}
Considerando un entorno académico de estudio de sistemas operativos, justamente el ambiente para el cual se
propicia esta investigación, con conocimientos previos de C (el antecesor de C++) y fundamentos de sistemas
operativos; HaikuOS es un proyecto con nivel de complejidad media, debido a su arquictectura híbrida modular,
la cual no es muy común, y a la cantidad de componentes implementados, y la profundidad de los mismos (por
ejemplo, la interfaz gráfica y optimización para multimedia). Además de la herramientas utilizadas para su
construcción y portabilidad.

También se consideró para colocarlo en el nivel medio, que la accesibilidad para los estudiantes es buena,
ya que el proyecto es open source, aunque no cuenta con tantos estudios en torno a él como otros sistemas.
Su página web oficial es bastante completa, y el repositorio oficial está bien organizado. Sin embargo, no
posee una documentación formal tan detallada como otros sistemas operativos.