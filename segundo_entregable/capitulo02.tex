\chapter{Comparación técnica}

La tabla siguiente presenta una comparación técnica de varios sistemas operativos destacados, considerando aspectos clave como arquitectura, lenguaje de programación, tamaño del kernel, soporte de hardware, documentación y comunidad de usuarios y desarrolladores.
\begin{sidewaystable}[p]
\centering
\caption{Tabla comparativa resumida de sistemas operativos educativos y experimentales (formato horizontal)}
\label{tab:so-resumida-horizontal}
\begin{tabular}{|p{2.5cm}|p{3cm}|p{2.7cm}|p{3cm}|p{2.6cm}|p{2.3cm}|}
\hline
\textbf{S.O.} & \textbf{Arquitectura} & \textbf{Lenguaje} & \textbf{Documentación} & \textbf{Comunidad} & \textbf{Dificultad} \\ 
\hline

\textbf{MINIX} &
Microkernel &
C, ASM &
Amplia y académica &
Pequeña–media &
Baja \\ \hline

\textbf{XV6} &
Monolítico simple &
C, ASM &
Excelente (MIT) &
Grande en educación &
Muy baja \\ \hline

\textbf{Theseus} &
Cells (nano-core) &
Rust &
Limitada; académica &
Muy pequeña &
Muy alta \\ \hline

\textbf{Redox OS} &
Microkernel híbrido &
Rust &
Buena; activa &
Media–alta &
Media–alta \\ \hline

\textbf{FlexOS} &
Híbrida modular &
C/C++, Rust &
Avanzada; técnica &
Pequeña &
Alta \\ \hline

\textbf{LibrettOS} &
Multiserver + Library OS &
C/C++, ASM &
Académica avanzada &
Muy pequeña &
Alta \\ \hline

\textbf{FreeRTOS} &
Microkernel RTOS &
C &
Excelente y abundante &
Muy grande &
Media \\ \hline

\textbf{HelenOS} &
Microkernel multiserver &
C/C++ &
Buena; técnica &
Media &
Media \\ \hline

\textbf{Haiku OS} &
Híbrido modular &
C++ &
Buena; enfocada &
Media &
Media \\ \hline

\textbf{FreeDOS} &
DOS real mode &
C, ASM &
Media; histórica &
Media &
Baja \\ \hline

\end{tabular}
\end{sidewaystable}

