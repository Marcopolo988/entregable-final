\section{FlexOS}

\subsection{Nombre del proyecto o sistema operativo}
El sistema operativo analizado es ``FlexOS''.  
Según \citep{flexos2023paper}, su nombre proviene del término “Flexible Operating System”, reflejando su principal objetivo: ofrecer una plataforma operativa flexible y configurable que permita ajustar dinámicamente el nivel de aislamiento entre componentes.  
A diferencia de los sistemas tradicionales con arquitecturas rígidas (monolíticas o microkernel), FlexOS propone una arquitectura adaptable donde el desarrollador puede equilibrar seguridad, aislamiento y rendimiento según las necesidades del sistema.

\subsection{Enlace al repositorio y/o documentación oficial}
\begin{itemize}
    \item \textbf{Repositorio oficial:} \url{https://github.com/project-flexos}
    \item \textbf{Sitio y documentación técnica:} \url{https://project-flexos.github.io/}
    \item \textbf{Publicación principal:} \textit{“FlexOS: Making OS Isolation Flexible”}, presentada en EuroSys 2023 \citep{flexos2023paper}.
\end{itemize}

\subsection{Objetivo del proyecto}
El objetivo central de FlexOS (educativo, experimental) es redefinir el equilibrio entre rendimiento y aislamiento en los sistemas operativos.  
Mientras los sistemas tradicionales deben optar entre un kernel monolítico rápido o un microkernel seguro pero más costoso en rendimiento, FlexOS permite ajustar el nivel de aislamiento de manera configurable.  
Esto se logra mediante su diseño modular, que permite que cada componente (como controladores, memoria o IPC) se ejecute con distintos niveles de separación temporal y espacial.  
De esta forma, se busca proporcionar un entorno operativo capaz de adaptarse a distintos contextos: desde sistemas embebidos con recursos limitados, hasta entornos de alta seguridad o investigación académica \citep{flexos2023paper}.

\subsection{Lenguaje(s) de implementación}
FlexOS está implementado principalmente en C y C++, con soporte parcial en Rust para módulos experimentales que requieren mayor seguridad en memoria.  
Su infraestructura está construida sobre el microkernel Unikraft, al que extiende con nuevas bibliotecas y mecanismos de aislamiento.  
La elección de C y C++ permite mantener compatibilidad con Unikraft y con componentes de sistemas POSIX existentes, mientras que Rust se emplea en zonas críticas de seguridad o donde la verificación de memoria resulta esencial \citep{flexos2023paper}.

\subsection{Arquitectura del sistema}
FlexOS presenta una arquitectura modular y configurable, inspirada en los principios de los microkernels, pero con un grado de flexibilidad superior.  
Cada componente puede ejecutarse en un dominio de aislamiento, que define si comparte espacio de direcciones, hilos de ejecución o recursos temporales con otros módulos \citep{flexos2023paper}.  
Estos dominios pueden configurarse de tres maneras principales:

El núcleo de FlexOS gestiona la coordinación de estos dominios, la planificación del tiempo de CPU, la comunicación entre módulos (IPC) y el control de errores.  
\begin{figure}[H]
\centering
\includegraphics[width=1\textwidth]{figures/aflexos.png}
\centering
\caption{Arquitectura modular e híbrida de FlexOS.}
\label{fig:flexos_architecture}
\centering
\small Fuente: Obtenido de \citep{flexos2023paper}.
\end{figure}


\subsection{Componentes implementados}

En FlexOS, los componentes del sistema se diseñan para maximizar el aislamiento y la modularidad, aplicando principios de sistemas operativos seguros y configurables.  
Su estructura permite ajustar el nivel de aislamiento (temporal y espacial) entre componentes según los requisitos de rendimiento o seguridad, convirtiéndolo en un sistema operativo híbrido altamente flexible.

Los componentes se implementan principalmente en C, C++ y Rust, utilizando microbibliotecas y servicios del kernel configurables.  
A continuación, se presenta una tabla con los principales subsistemas y módulos de FlexOS, junto con su función técnica principal \citep{flexos2023paper}.

\begin{table}[H]
\centering
\caption{Funcionamiento técnico de los principales subsistemas de FlexOS}
\begin{tabular}{|p{5cm}|p{4cm}|p{6cm}|}
\hline
\textbf{Componente / Sub-sistema} & \textbf{Nombre o módulo en FlexOS} & \textbf{Funcionamiento técnico (idea clave)} \\ \hline
Núcleo y planificación & \texttt{flex\_kernel} & Controla la ejecución y distribución del tiempo de CPU entre tareas. \\ \hline
Gestión de memoria & \texttt{mem\_mgr}, \texttt{isolation\_lib} & Maneja memoria virtual y niveles configurables de aislamiento. \\ \hline
Procesos y tareas & \texttt{task\_mgr} & Crea y coordina tareas con comunicación segura. \\ \hline
Sistema de IPC & \texttt{ipc\_core}, \texttt{shared\_mem} & Permite intercambio de datos mediante colas y memoria compartida. \\ \hline
Seguridad y aislamiento & \texttt{sandbox}, \texttt{trusted\_domain} & Define dominios seguros y políticas de acceso. \\ \hline
Gestión de E/S y drivers & \texttt{io\_layer}, \texttt{device\_mgr} & Ejecuta controladores aislados para proteger el kernel. \\ \hline
Monitoreo y fallos & \texttt{fault\_handler}, \texttt{recovery\_svc} & Detecta errores y reinicia módulos sin afectar el sistema. \\ \hline
Configuración del sistema & \texttt{flex\_config} & Ajusta el nivel de aislamiento y rendimiento. \\ \hline
Bibliotecas de usuario & \texttt{libflex}, \texttt{api\_runtime} & Facilita la comunicación segura con el kernel. \\ \hline
\end{tabular}
\centering
Fuente: Elaboración propia con base en \citep{flexos2023paper}.
\end{table}


\subsubsection*{Gestión de memoria}
FlexOS implementa un modelo de gestión de memoria basado en \textit{espacial isolation}, donde cada dominio o componente puede tener su propio espacio de direcciones.  
El sistema utiliza tanto protección por hardware (MMU) como políticas de software para garantizar independencia entre procesos y minimizar los efectos de fallos de memoria.

\subsubsection*{Procesos y multitarea}
El administrador de tareas controla el ciclo de vida de los procesos mediante colas de ejecución.  
Los procesos pueden compartir recursos limitadamente o ejecutarse en dominios completamente aislados, dependiendo de la configuración elegida.

\subsubsection*{Comunicación entre módulos}
FlexOS utiliza un sistema híbrido de comunicación: colas seguras, llamadas IPC síncronas y memoria compartida protegida.  
Esto permite un equilibrio entre rendimiento y seguridad, ajustable según la política de aislamiento establecida.

\subsubsection*{Gestión de E/S y drivers}
Los controladores en FlexOS se ejecutan como módulos independientes o en dominios restringidos, minimizando el riesgo de fallos.  
Cada driver puede ser reiniciado o reemplazado sin afectar el núcleo principal.

\subsubsection*{Seguridad y aislamiento}
Una de las características clave del sistema es su capacidad para reconfigurar el aislamiento entre componentes, ofreciendo niveles ajustables de separación espacial y temporal.  
Esto permite ejecutar aplicaciones críticas en entornos seguros sin comprometer el rendimiento global del sistema.


\subsection{Herramientas utilizadas (compiladores, emuladores, etc.)}

FlexOS se construye sobre la infraestructura de \textbf{Unikraft}, por lo que emplea un conjunto de herramientas especializadas para la compilación, configuración y ejecución de sus módulos.  
El proceso de construcción del sistema operativo se basa en la cadena de herramientas de \texttt{GNU Make} y \texttt{Kconfig}, lo que permite personalizar la inclusión o exclusión de componentes del kernel según el nivel de aislamiento deseado.  

Entre las herramientas más relevantes utilizadas en FlexOS se encuentran:

\begin{itemize}
    \item \textbf{Compilador principal:} \texttt{clang/LLVM}, utilizado para la instrumentación del código, generación de binarios seguros y soporte para aislamiento entre dominios.
    \item \textbf{Infraestructura de construcción:} \texttt{Unikraft Build System}, que proporciona un entorno modular basado en \texttt{KBuild} y \texttt{Makefiles} para definir configuraciones del kernel y librerías.
    \item \textbf{Simuladores y entornos de prueba:} ejecución de instancias de FlexOS sobre \texttt{QEMU} y máquinas virtuales \texttt{KVM}, permitiendo validar configuraciones con diferentes niveles de aislamiento y medir overheads de seguridad.
    \item \textbf{Instrumentación y análisis:} uso de \texttt{AddressSanitizer (ASan)}, \texttt{Clang Control Flow Integrity (CFI)} y trazadores de rendimiento para detectar vulnerabilidades y medir costos de comunicación entre dominios.
    \item \textbf{Herramientas de fuzzing y validación:} integración con \texttt{ConfFuzz}, el framework de fuzzing desarrollado por los mismos autores, usado para explorar automáticamente configuraciones de aislamiento y detectar fallos en el espacio de diseño.
\end{itemize}

\subsection{Nivel de complejidad y accesibilidad para estudiantes}

FlexOS presenta un nivel de complejidad intermedio a avanzado, dado su enfoque experimental y su dependencia de la infraestructura de Unikraft y del compilador LLVM.  
Aunque está diseñado como un sistema de investigación y enseñanza sobre \textit{microkernels híbridos y aislamiento flexible}, su curva de aprendizaje es considerablemente mayor que la de sistemas educativos tradicionales como XV6 o MINIX.

Su accesibilidad depende del objetivo académico:

\begin{itemize}
    \item Para estudiantes de pregrado, el entorno de FlexOS puede resultar complejo por su uso intensivo de compilación cruzada, configuración Kconfig y dependencias de LLVM, pero es útil para entender conceptos de modularidad y aislamiento.
    \item Para estudiantes de posgrado o investigadores, ofrece un entorno ideal para estudiar la relación entre \textit{rendimiento y seguridad}, mediante experimentación con distintos niveles de compartmentalización, IPC y aislamiento de memoria.
\end{itemize}

