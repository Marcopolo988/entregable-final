\chapter*{Introducción}
\addcontentsline{toc}{chapter}{Introducción}

Para desarrollar un sistema operativo desde sus bases es indispensable conocer, con suficiente detalle, las distintas estructuras y mecanismos que hacen posible su funcionamiento. Más allá de la teoría, resulta especialmente útil examinar sistemas operativos reales —muchos de ellos abiertos y creados con fines educativos o de investigación— que exploran diferentes arquitecturas, modelos de aislamiento y niveles de complejidad.

En este informe se estudian diversas propuestas existentes, entre ellas MINIX, XV6, Theseus, RedoxOS, FlexOS, LibrettOS, FreeRTOS, HelenOS, Haiku OS y FreeDOS. Lo que se busca es aprender de las decisiones de diseño que presentan estos proyectos, contrastar sus propuestas y ver cuáles resultan realmente útiles como referencia para construir un sistema operativo con fines académicos.

\section*{Objetivo de la investigación}

La idea central del estudio es revisar y comparar técnicamente varios sistemas operativos que responden a diferentes arquitecturas y enfoques de diseño, prestando particular atención a los proyectos de código abierto que se usan para enseñar o investigar.

\section*{Alcance del análisis}

Este estudio se limita a sistemas operativos que, por su licencia, documentación y propósito, pueden emplearse adecuadamente como referencia en la construcción de un sistema operativo académico. En particular, se revisan los siguientes proyectos: MINIX, XV6, Theseus, RedoxOS, FlexOS, LibrettOS, FreeRTOS, HelenOS, Haiku OS y FreeDOS, cada uno abordado en el Capítulo 1 bajo un esquema de análisis uniforme.

El alcance comprende únicamente:

\begin{enumerate}
\item La revisión de documentación oficial, artículos académicos y repositorios de código.
\item Analizar cualitativamente las decisiones de diseño que se tomaron en cada uno de los sistemas revisados.
\item Elaborar una comparación organizada y un análisis crítico que permita reconocer cuáles de estas propuestas pueden servir mejor como referencia para desarrollar un sistema operativo con fines académicos.
\end{enumerate}

\section*{Metodología}

Para cumplir con los objetivos planteados, se adopta una metodología dividida en tres fases principales:

\begin{enumerate}
\item \textbf{Revisión bibliográfica y documental:} Aquí se reúne y examina información procedente de libros especializados, artículos académicos, documentación oficial de los proyectos y repositorios de código. Gracias a esta revisión es posible entender el contexto, los objetivos iniciales y la situación actual de cada uno de los sistemas operativos analizados

\item \textbf{Evaluación técnica de cada sistema operativo:} En el Capítulo 1 se estudia cada sistema usando un mismo esquema. Se revisa para qué sirve, cómo está diseñado, qué partes tiene (procesos, memoria, archivos, dispositivos, etc.), qué herramientas se utilizan para trabajarlo, como compiladores o emuladores.y qué tan difícil o accesible es para los estudiantes.

\item \textbf{Comparación estructurada y análisis:} Con base en la información recopilada, el Capítulo 2 desarrolla una comparación general entre los sistemas analizados, evaluando criterios como arquitectura, tamaño del kernel, soporte de hardware, comunidad, documentación y su potencial para fines formativos. En el Capítulo 3 se presenta un análisis crítico donde se examinan las ventajas y limitaciones  de cada propuesta, reflexionando sobre su idoneidad como punto de partida para la construcción de un sistema operativo.
\end{enumerate}
